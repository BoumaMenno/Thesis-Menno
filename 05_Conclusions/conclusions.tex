\documentclass[../DC2019003Bouma.tex]{subfiles}
\begin{document}
\graphicspath{{05_Conclusions/img/}}
\renewcommand{\chaptermark}[1]{\markboth{\thechapter.\ #1}{}}
\renewcommand{\sectionmark}[1]{\markright{#1}{}}
\pagestyle{fancyreport}
\cleartooddpage
\pagestyle{fancyreport}
\chapter{Conclusions and Recommendations}\label{ch:concl}
This chapter concludes this research and contains recommendations for future research. As stated in Section~\ref{sec:1resobj}, the research objective of this work is twofold. Firstly, we aim at finding a model suitable for mechanical systems with unilateral constraints and spatial friction, and extend the sensitivity analysis presented in \cite{Rijnen2018a} to be suitable for such models. Secondly, in the cases of simultaneous releases, this work aims at extending the sensitivity analysis to be suitable for trajectories with input-triggered events. These research objectives have been formally defined as:\\\\
\textbf{Simultaneous impacts with friction}\\
\textit{Find a model suitable to describe mechanical systems with unilateral constraints and spatial friction. The sensitivity analysis and mathematical notation presented in \cite{Rijnen2018a} shall be extended to be compatible with such models.}

\textbf{Simultaneous releases}\\
\textit{Extending the sensitivity analysis presented in \cite{Rijnen2018a} such that it is suitable for input-triggered events.}

In the following sections, the conclusions and recommendations related to these research objectives are given. In Section~\ref{sec:6con}, the most important conclusions are drawn and the contribution of this work is summarized. Then, in Section~\ref{sec:6rec}, a number of suggestions for future research is given.

\section{Conclusions}\label{sec:6con}
In this section, the conclusions that are drawn from this work are presented. The conclusions are subdivided in several subjects, which correspond with the outline of the chapters in this work. Each subject is concluded by the contribution associated with that subject, given in \textit{italic}. Together, these contributions achieve the research objective first presented in Section~\ref{sec:1resobj}.

\textbf{Modeling.} To analyze mechanical systems with unilateral constraints and spatial friction, a suitable model needed to be found. The hybrid system formulation with guard functions has been previously used in modeling mechanical systems with trajectories experiencing impact \cite{Rijnen2018a}. In Chapter~\ref{ch:model}, a hybrid system formulation using guard functions is presented for mechanical systems with trajectories experiencing frictional impacts, releasing motions, and stick/slip transitions. We name this model the nonlinear state-and-input-triggered hybrid system (NSITHS).

\textit{A model, suitable to describe mechanical systems with unilateral constraints and spatial friction, is found and presented in the form of an NSITHS.}

\textbf{State-and-input-triggered isolated events.} The model presented in Chapter~\ref{ch:model} is used to analyze the effect of perturbations on trajectories of mechanical systems with unilateral constraints and spatial friction. By performing a sensitivity analysis, a model that describes the response of the system on perturbations is obtained. Since the model presented in Chapter~\ref{ch:model} contains input-dependent guard functions, the sensitivity analysis presented in \cite{Rijnen2017} does not suffice. Therefore, the sensitivity analysis is extended to be suitable for input-dependent guard functions and jump maps. This sensitivity analysis defines a linear time-triggered hybrid system (LTTHS), which approximates the behavior of the NSITHS for trajectories with isolated events.

\textit{The sensitivity analysis and notation presented in \cite{Rijnen2018a} is extended to be compatible with the NSITHS, for trajectories with isolated state-and-input-triggered events.}

\textbf{State-and-input-triggered simultaneous events.} The approximation of the NSITHS presented in Chapter~\ref{ch:order}, named the LTTHS, is only valid for trajectories with isolated events. In Chapter~\ref{ch:simult}, the LTTHS is extended to be suitable for nominal trajectories with simultaneous events. Using the approach presented in \cite{Rijnen2018}, the jump gain that describes the jumping behavior in the LTTHS is replaced by the positively homogeneous jump gain. This positively homogeneous jump gain approximates the jumping behavior of the trajectories around the nominal trajectory, where loss of simultaneity occurs. Using the positively homogeneous jump gain, the positively homogeneous time-triggered hybrid system (PTTHS) is defined. The PTTHS approximates the behavior of the NSITHS for trajectories with simultaneous events. The work in \cite{Rijnen2018a} presents the PTTHS for state-triggered events and state-dependent jump maps, whereas this work presents the PTTHS for state-and-input-triggered events and state-and-input dependent jump gains.

\textit{The sensitivity analysis and notation presented in \cite{Rijnen2018a} is extended to be compatible with the NSITHS, for trajectories with simultaneous state-and-input-triggered events.}

To summarize, this work presents an analysis on the tracking behavior of mechanical systems with unilateral constraints and spatial friction. The analysis is compatible with both state-and-input-dependent guard functions and jump maps, and a class of nominal trajectories that can experience both isolated and simultaneous events. The reader should be aware that the theory is not valid for near-simultaneous events in the nominal trajectory. The analysis results in an approximation of trajectories close to the nominal trajectory, which we claim can be used to assess the local asymptotic stability of the original system. Since the approximation is time-triggered, conventional stability analysis tools can be used to assess the stability of the state-and-input triggered system. Also, the approximation can be used for LQR optimal control settings, or even receding horizon control settings.

\section{Recommendations}\label{sec:6rec}
Several research objectives have been addressed in this work. Since research is performed in a particular scope, this work has limitations which are translated into suggestions for future research. Also, due to the continuously progressing nature of research, new potential research subjects have arisen during the writing of this work. In this section, a number of suggestions is made for future research. We presume these suggestions to contribute to the methodology of reference spreading (RS) and the search for a high performance tracking control strategy for mechanical systems physical interaction at non-negligible speed. The recommendations are presented in no particular order.

\textbf{Numerical validation.} The contribution of this work lies in extending the work in \cite{Rijnen2018a} to be compatible with state-and-input-dependent guards, and needs to be validated by means of numerical simulation. While first-steps have been taken in performing this validation, the scope of this project did now allow for a complete validation to be finished. A numerical simulation for mechanical systems with state-triggered events is already available in a planar setting, which can be extended to be suitable for state-and-input-triggered events by implementing releasing motions. When this addition has been made, the next step can be made by including dry friction, resulting in stick/slip transitions and frictional impacts. There is an unsolved problem with the guard function associated with slip-to-stick transitions, which can lead to numerical problems. A solution should be found for this guard function assess stick/slip transitions in the numerical validations. Finally, these simulations should be extended to a 3-dimensional settings, and systems of higher complexity.

\textbf{General definition positively homogeneous jump gain.} One of the main contributions of this is work is the extension of the positively homogeneous to input-dependent guards. While this positively homogeneous jump gain can be defined for any event character, this work solely focuses on character-2 events. The RS control approach would benefit from a general definition of the positively homogeneous jump gain, which can be used to straightforwardly compute the positively homogeneous jump gain for an event of arbitrary character.

\textbf{Stick/slip impacts.} With the goal of designing a high performance control strategy for mechanical systems with unilateral constraints and spatial friction, this work has led to another suggestion for future research. When a contact point impacts a surface, the post-event state of that event could immediately activate a stick/slip-transition related guard function. For a high performance control strategy, we often desire to act on the border between stick and slip, e.g., during a running motion of a robot. Currently, the theory is not applicable to situations where the post-event state of an event immediately activates another guard function. We suggest that future research should entail find a jump gain that can approximate the behavior of such events, under the presence of perturbations.

\textbf{Trajectory generation.} To apply the RS control approach, a nominal trajectory that satisfies the posed assumptions is required. This reference trajectory has proven to be not straightforward to find. We believe that finding a straightforward method for defining this reference trajectory, along with its extensions, is a valuable addition to the RS control approach. The extensions are currently designed in a non-generalizable way. A general algorithm that generates the extensions can lead to improvements in domain of attraction, robustness, and convergence rate. Another approach can be to generate the reference trajectories by the machine learning approach `learning by demonstration'.

\textbf{Mathematical accuracy.} This work can benefit from more extensive research in the mathematical accuracy of the sensitivity analysis. The stability of the LTTHS is found to imply local asymptotic stability of the associated nonlinear state-triggered hybrid system (NSTHS) in \cite{Rijnen2017}. While we believe it exists, this relation is not yet proven for the LTTHS derived from the NSITHS. Therefore, we suggest investigating this relation as future research. Correspondingly, we believe the same relation can be found between the PTTHS and the NSTHS, and the PTTHS and the NSITHS. In addition, we belief the approximations defined by the LTTHS and PTTHS are of first order. We expect that further investigation in the accuracy of these approximations will lead to a proof for the approximations being first-order. Finally, the jump gains $\Gb$ and $\Jb$ have only been formally derived for character-2 events. We claim that, using an induction-like proof, the derivation of the jump-gain can be extended to events of an arbitrary character.
\end{document}