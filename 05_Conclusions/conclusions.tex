\documentclass[../DC2017114Bouma.tex]{subfiles}
\begin{document}
\graphicspath{{05_Conclusions/img/}}
\renewcommand{\chaptermark}[1]{\markboth{\thechapter.\ #1}{}}
\renewcommand{\sectionmark}[1]{\markright{#1}{}}
\pagestyle{fancyreport}
\cleartooddpage
\pagestyle{fancyreport}
\chapter{Conclusions and Recommendations}\label{ch:concl}
The research goals that have been addressed in this work are:\\\\
\textbf{Simultaneous impacts with friction}\\
\textit{Find a model suitable to describe mechanical systems with unilateral constraints and spatial friction. The sensitivity analysis and mathematical notation presented in \cite{Rijnen2018a} shall be extended to be compatible with such models.}

\textbf{Simultaneous releases}\\
\textit{Extending the sensitivity analysis presented in \cite{Rijnen2018a} such that it is suitable for input-triggered events.}

\section{Conclusions}
To analyze mechanical systems with unilateral constraints and spatial friction, a suitable model needed to be found. The hybrid system formulation using guard functions has been previously used in modeling mechanical systems with trajectories experiencing impact \cite{Rijnen2018a}. In Chapter~\ref{ch:model}, a hybrid system formulation using guard functions is presented for mechanical systems with trajectories experiencing frictional impacts, releasing motions, and stick-slip transitions. We name this model the nonlinear state-and-input-triggered hybrid system (NSITHS). The contribution made in Chapter~\ref{ch:model} is:

\textit{A model, suitable to describe mechanical systems with unilateral constraints and spatial friction, is found and presented in the form an NSITHS.}

The model presented in Chapter~\ref{ch:model} is used to analyze the effect of perturbations on trajectories of mechanical systems with unilateral constraints and spatial friction. This is achieved by performing a sensitivity analysis, which results in a model that describes the response on perturbations of the system. Since the model presented in Chapter~\ref{ch:model} contains input-dependent guard functions, the sensitivity analysis presented in \cite{Rijnen2017} does not suffice. Therefore, the sensitivity analysis is extended to be suitable for input-dependent guard functions and jump maps. This sensitivity analysis defines a linear time-triggered hybrid system (LTTHS), which approximates the behavior of the NSITHS for trajectories with isolated events. The contribution made in Chapter~\ref{ch:order} is:

\textit{The sensitivity analysis and notation presented in \cite{Rijnen2018a} is extended to be compatible with the NSITHS, for trajectories with isolated events.}


\begin{itemize}
\item PTTHS for systems with input dependent guards
\item First steps in simulations
\item Summary of contribution can accomplish
\end{itemize}

\section{Recommendations}
\begin{itemize}
\item Trajectory generation
\item Numerical Validation (also slip to stick guard function suitable for simulations)
\item Missing proofs (Induction proof of $\Gb$ and $\Jb$, stability of LTTHS/PTTHS implies stability of NSITHS, approximation is first order)
\item Impacts at the border between stick and slip
\end{itemize}
\end{document}