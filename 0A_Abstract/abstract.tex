\documentclass[../DC2019003Bouma.tex]{subfiles}
\begin{document}
\cleartooddpage
\pagenumbering{roman}
\chapter*{Abstract}\phantomsection\addcontentsline{toc}{chapter}{Abstract}
Everyday we touch, grab, or throw objects in our environment. This interaction with our environment comes natural to us; we do this without actively thinking about it. This is not the case for robots. To avoid complexity of the controller, many solutions for physically interacting robots require the contact to happen at zero relative velocity. Such movements, however, are very unnatural and hinder the robot in situations where speed is of significance, e.g., an industrial robot aiming to reach a certain throughput or a quadruped performing running motions. This thesis works towards a control strategy capable of handling contacts at non-negligible speed, increasing the performance of physically interacting robots and achieving more fluent, natural motions.

\end{document}