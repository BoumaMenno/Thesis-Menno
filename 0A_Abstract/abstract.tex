\documentclass[../DC2019003Bouma.tex]{subfiles}
\begin{document}
\cleartooddpage
\pagenumbering{roman}
\chapter*{Abstract}\phantomsection\addcontentsline{toc}{chapter}{Abstract}
Physical interaction with our environment is something we do everyday. We type on our keyboards, open doors, with every step we take we are pushing ourselves away from the ground beneath our feet. Most of our daily activities involve physical interaction with the objects around us. This interaction comes natural to us; we do this without actively thinking about it. This is not the case for robots. To avoid complexity of the controller, many solutions for physically interacting robots require the contact to happen at zero relative velocity. Such movements, however, are very unnatural and hinder the robot in situations where speed is of significance, e.g., an industrial robot aiming to reach a certain throughput or a quadruped performing running motions. This thesis works towards a control strategy capable of handling contacts at non-negligible speed, increasing the performance of physically interacting robots and achieving more fluent, natural motions.

The main difficulty in tracking of trajectories with state jumps, is that the jumping behavior of the tracking trajectory differs from the jumping behavior of the reference trajectory. The event times, ante-event state, and in some cases, number of jumps do not coincide. This work contributes to a sensitivity analysis for trajectories with; state jumps, which analyzes the behavior of trajectories close to the reference. The proposed method is named reference spreading, as it is based on a notion of error where the reference trajectories are extended past their nominal jump times. Using these extensions, the jumping behavior of the trajectories close to the reference trajectory is approximated, giving insight in the tracking behavior of the system and allowing for optimal tuning of the feedback gains to achieve tracking. 

This work focuses on expanding the class of trajectories considered by the reference spreading methodology, by assessing systems experiencing spatial friction and considering trajectories with releasing motions. The current state of the sensitivity analysis considers position-triggered events that generate impulsive forces normal to the impacting surface. The class of trajectories with spatial friction and releasing motions are defined by events that are triggered on velocity or acceleration level, and introduces events with both normal and tangential impulsive forces. This work poses a solution to the difficulties that this class of trajectories brings along.


\textbf{Keywords:} \quad trajectory tracking, impact, mechanical system, unilateral constraint, spatial friction, hybrid system, release, state-and-input-triggered event, simultaneous event, sensitivity analysis
\end{document}