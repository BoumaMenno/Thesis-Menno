%% Package enlightening:
%% a4wide       for make the vertical margins smaller
%% europs       for typing the euro symbol
%% graphicx     for inserting figures,graphs,photo's
%% fancyhdr     for creating a prettier header/footer
%% moreverb     for inserting source code better
%% subfigure    for allowing multiple figures/tables in one enviroment
%% float        Place figures HERE with the [H] option
%% amsmath      for more math
%% booktabs     for nicer tables (\toprule, etc)
%% babel        for "english" hyphenation and other textual styles.
%% hyperref     for creating clickable links and the bookmarksmenu
%% ifpdf        for making the code latex and pdflatex compatible
%% psfrag       for replacing text in eps graphics
%% pdfrack      for replacing text in pdf graphics "bash needed!"
%% caption      for changing the properties (font) of the caption
%% setspace     reduce linespacing in the TOC or Bib \begin{spacing}{0.9} ...
%% listings     for formatting Matlab code
%% lipsum       for creating dummy text (Lorem ipsum dolor sit amet...)
%% soul         for highlighting text \hl{}
%% xspace       for putting or removing a space for \newcommand macro's
%% caption      for changing font size and alignment of captions
\documentclass[11pt,a4paper]{report}

%%\usepackage{a4wide,moreverb,ifpdf}
\usepackage{subfiles}
\usepackage{textcomp}
\usepackage{hyperref}
\usepackage{amsmath,booktabs,cite,amssymb,bm,soul}
\usepackage{amsthm}
\usepackage{mathtools}
\usepackage{esvect}
\usepackage{fancyhdr}
\usepackage{float}
\usepackage{url}
\usepackage[right=2.2cm,left=2.2cm,top=3.2cm,bottom=3.0cm]{geometry}
%\usepackage[FIGBOTCAP]{subfigure}
%\usepackage[dutch]{babel}
%\usepackage[scriptsize,sf,hang]{caption}
\usepackage[usenames,dvipsnames]{color}
\usepackage[pdftex]{graphicx}
\usepackage{enumitem}
\usepackage{caption}
\usepackage{subcaption}
\usepackage{chngpage}
\usepackage{chngcntr}
\usepackage{pdflscape}
\usepackage[version=3]{mhchem}
\usepackage{epstopdf}
\usepackage{dsfont}
\usepackage{appendix}
\usepackage[textfont=it,figurewithin=section,tablewithin=section]{caption}
%\sethlcolor{red}       % text highlight color red (soul package)
\usepackage{rotating}  % rotate mechanical drawings with sidewaysfigure
%\usepackage{fixltx2e} % for sub- and superscript
%\usepackage{picins}    % be able to put picture next to text
\usepackage{array}     % extended functions for tables
\usepackage{wrapfig}   % be able to put picture next to text
\usepackage{xspace}    % check if a space is needed after \newcommand
\usepackage[absolute]{textpos}

\usepackage{wasysym}

\usepackage{pdfpages}

% generate nomenclature
\usepackage{nomencl}
\makenomenclature

\usepackage{xstring}
\usepackage{xstring}
\usepackage{xpatch}
\patchcmd{\thenomenclature}
  {\leftmargin\labelwidth}
  {\leftmargin\labelwidth\itemindent 1em }
  {}{}
\newcommand{\nomenclheader}[1]{%
  \item[\hspace*{-\itemindent}\normalfont\bfseries#1]}
\renewcommand\nomgroup[1]{%
  \IfStrEqCase{#1}{%
   {G}{\nomenclheader{Greek symbols}}% G - Greek symbols
   {R}{\nomenclheader{Roman symbols}}% R - Roman symbols
   {A}{\nomenclheader{Abbreviations}}% A - Abbreviations
   {S}{\nomenclheader{Sub- and superscripts}}% S - Sub- and superscripts
  }%
} 
\setlength{\nomitemsep}{1pt}
\nomlabelwidth=30mm

% textpos package arguments
\setlength{\TPHorizModule}{1mm}
\setlength{\TPVertModule}{\TPHorizModule}
% use option [showboxes] to package textpos to draw the box
\definecolor{myblue}{rgb}{0,0,0.4}

% define Definition, Assumptions, and Remarks
\newtheorem{mydef}{Definition}
\newtheorem{myass}{Assumption}
\newtheorem*{myremark}{Remark}
%% fixing the standard latex indentation style on new alinea.
\setlength{\parskip}{1ex}
\setlength{\parindent}{0mm}

%% extra document related shortcuts
\newcommand{\degree}{\ensuremath{^{\circ}}\xspace}
\newcommand{\HRule}{\rule{--\linewidth}{0.5mm}}

% Empty page when chapter starts at even page
\newcommand\cleartooddpage{\clearpage\ifodd\value{page}\else\null\thispagestyle{empty}\clearpage\fi}

% Remove counter resets at section level
\counterwithout{equation}{section}
\counterwithout{figure}{section}
\counterwithout{table}{section}

% Change numbering format to chapter level
\numberwithin{equation}{chapter}
\numberwithin{figure}{chapter}
\numberwithin{table}{chapter}

\definecolor{dkgreen}{rgb}{0,0.6,0}
\definecolor{gray}{rgb}{0.5,0.5,0.5}
\definecolor{mauve}{rgb}{0.58,0,0.82}

\usepackage{listings}
\lstset{
    language=Matlab,
    basicstyle=\color{black}\ttfamily\footnotesize,
    numbers=left,
    numberstyle=\tiny\color{gray},    
    stepnumber=1,                 
    numbersep=5pt,               
    commentstyle=\color{dkgreen}\itshape\ttfamily,
    keywordstyle=\color{black}\ttfamily\footnotesize,
    showstringspaces=false,
    frame=single,
    backgroundcolor=\color{white},
    keywordstyle=\color{blue}
}

\usepackage{epsfig}   % To include Matlab figures
\usepackage{graphicx} % To be able to edit graphics
\usepackage{multirow} % to be able to use merged cells in tables
\usepackage[latin1]{inputenc} % to be able to type things like é, ä, ù and ê.
\usepackage{cite}

% Nicer lists
\newenvironment{packed_item}{
\begin{itemize}
  \vspace*{0em}
  \setlength{\itemsep}{0pt}
  \setlength{\parskip}{0pt}
  \setlength{\parsep}{0pt}
}{\end{itemize}}

% creating the fancy header
\renewcommand{\chaptermark}[1]{\markboth{\thechapter.\ #1}{}}
\renewcommand{\sectionmark}[1]{\markright{#1}{}}

\fancypagestyle{fancyreport}{
\fancyhf{} % remove everything
\lhead{\small \leftmark}
\rhead{\small }
%\chead{\small }
\lfoot{\small \rightmark}
\rfoot{\small \thepage}
\cfoot{\small}
\renewcommand{\headrulewidth}{1pt}
\renewcommand{\footrulewidth}{1pt}
}

\fancypagestyle{appendix}{%
\fancyhf{} % remove everything
\lhead{\small \leftmark}
\rhead{\small }
%\chead{\small }
\lfoot{\small \rightmark}
\rfoot{\small \thepage}
\cfoot{\small}
\renewcommand{\headrulewidth}{1pt}
\renewcommand{\footrulewidth}{1pt}
}

\fancypagestyle{plain}{ %
\fancyhf{} % remove everything
%\chead{\small }
\lfoot{\small }
\rfoot{\small \thepage}
\cfoot{\small}
\renewcommand{\headrulewidth}{0pt} % remove lines as well
\renewcommand{\footrulewidth}{1pt}
}
% Break rules in Matlab code
\lstset{breaklines=true}

\newcommand{\nocontentsline}[3]{}                                                   %Zorgt ervoor dat je met het command \tocless kunt zorgen dat bijv. een section
\newcommand{\tocless}[2]{\bgroup\let\addcontentsline=\nocontentsline#1{#2}\egroup}  %niet in de TableOfContent wordt opgenomen, maar de nummerig wel doorloopt. 
\setcounter{tocdepth}{2} %Geen subsections in table of contents
%\AtBeginDocument{\addtocontents{toc}{\protect\thispagestyle{plain}}} 

\newcommand{\R}{\mathbb{R}}

\newcommand{\ab}{{\boldsymbol{a}}}
\newcommand{\bb}{{\boldsymbol{b}}}
\newcommand{\eb}{{\boldsymbol{e}}}
\newcommand{\fb}{{\boldsymbol{f}}}
\newcommand{\zb}{{\boldsymbol{z}}}
\newcommand{\gb}{{\boldsymbol{g}}}
\newcommand{\hb}{{\boldsymbol{h}}}
\newcommand{\qb}{{\boldsymbol{q}}}
\newcommand{\rb}{{\boldsymbol{r}}} 
\newcommand{\ub}{{\boldsymbol{u}}}
\newcommand{\vb}{{\boldsymbol{v}}}
\newcommand{\wb}{{\boldsymbol{w}}}
\newcommand{\xb}{{\boldsymbol{x}}}
\newcommand{\yb}{{\boldsymbol{y}}}
\newcommand{\pb}{{\boldsymbol{p}}}

\newcommand{\Ab}{{\boldsymbol{A}}}
\newcommand{\Bb}{{\boldsymbol{B}}} 
\newcommand{\Cb}{{\boldsymbol{C}}}
\newcommand{\Db}{{\boldsymbol{D}}}
\newcommand{\Gb}{{\boldsymbol{G}}} 
\newcommand{\Hb}{{\boldsymbol{H}}} 
\newcommand{\Eb}{{\boldsymbol{E}}} 
\newcommand{\Ib}{{\boldsymbol{I}}} 
\newcommand{\Kb}{{\boldsymbol{K}}}
\newcommand{\Lb}{{\boldsymbol{L}}} 
\newcommand{\Mb}{{\boldsymbol{M}}}  
\newcommand{\Nb}{{\boldsymbol{N}}} 
\newcommand{\Pb}{{\boldsymbol{P}}} 
\newcommand{\Qb}{{\boldsymbol{Q}}}  
\newcommand{\Sb}{{\boldsymbol{S}}}  
\newcommand{\Tb}{{\boldsymbol{T}}} 
\newcommand{\Wb}{{\boldsymbol{W}}}  
\newcommand{\Jb}{{\boldsymbol{J}}} 
\newcommand{\Fb}{{\boldsymbol{F}}} 
\newcommand{\Yb}{{\boldsymbol{Y}}}
\newcommand{\Zb}{{\boldsymbol{Z}}}
\newcommand{\Zerob}{{\boldsymbol{0}}}

\newcommand{\Rbb}{\mathbb{R}}

\newcommand{\kappab}{\boldsymbol{\kappa}}  
\newcommand{\gammab}{\boldsymbol{\gamma}}  
\newcommand{\xib}{\boldsymbol{\xi}}
\newcommand{\phib}{\boldsymbol{\phi}}  
\newcommand{\lambdab}{\boldsymbol{\lambda}}
\newcommand{\rhob}{\boldsymbol{\rho}}  
\newcommand{\zetab}{\boldsymbol{\zeta}}
\newcommand{\alphab}{\boldsymbol{\alpha}}
\newcommand{\mub}{\boldsymbol{\mu}}
\newcommand{\nub}{\boldsymbol{\nu}}

\newcommand{\Lambdab}{\boldsymbol{\Lambda}}
\newcommand{\Sigmab}{\boldsymbol{\Sigma}}
\newcommand{\Gammab}{\boldsymbol{\Gamma}}

\newcommand{\Phib}{\boldsymbol{\Phi}} % Constr. Config. set 

\newcommand{\hrm}{\mathrm{h}}
\newcommand{\hbf}{\mathbf{h}}
\newcommand{\vrm}{\mathrm{v}}
\newcommand{\vbf}{\mathbf{v}}
\newcommand{\ubf}{\mathbf{u}}

\newcommand{\Ic}{\mathcal{I}} % closed contacts set
\newcommand{\Qc}{\mathcal{Q}} % configuration manifold
\newcommand{\Cc}{\mathcal{C}} % constrained manifold
\newcommand{\cc}{\mathit{c}} % constrained manifold
\newcommand{\Dc}{\mathcal{D}} % jump set
\newcommand{\dc}{\mathit{d}} % jump set
\newcommand{\Fc}{\mathcal{F}} % flow map
\newcommand{\fc}{\mathit{f}} % flow map
\newcommand{\Gc}{\mathcal{G}} % jump map
\newcommand{\gc}{\mathit{g}} % jump map
\newcommand{\ls}{\hspace{0cm}}

\newcommand{\Nc}{\mathcal{N}}
\newcommand{\rvec}{\vv{\boldsymbol{r}}}
\newcommand{\evec}{\vv{\boldsymbol{e}}}

\DeclareMathOperator{\prox}{\text{prox}}
\DeclareMathOperator{\dom}{\textnormal{dom}}
\DeclareMathOperator{\eve}{\textnormal{eve}}
\DeclareMathOperator{\Sign}{\textnormal{Sign}}
\DeclareMathOperator{\SgnSp}{\textnormal{SgnSp}}
\DeclareMathOperator{\argmax}{\text{arg\hspace{1pt}max}}
\DeclareMathOperator{\argmin}{\text{arg\hspace{1pt}min}}